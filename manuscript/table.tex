
        %TODO to repeat on a bigger machine to avoid vg crashing
    \begin{table}[h!]
        \caption{ Example of performance improvement between
            \cmd{odgi} and \cmd{vg}~\citep{vgtools} equivalent
            commands with output file sizes. The centromere region was
            extracted from CHR 6 of the HPRC year one assembly, i.e.,
            88~haploid, phased human genome assemblies from
            44~individuals plus the chm13 cell line and GRCh38
            reference genomes stored in 4.3GB~GFA format~\citep{GFA}.
            %
            First GFA was to converted to xg and next to \odgi\ and
            \vg\ formats respectively.  \odgi's path implementation
            can process paths in parallel, outperforming \vg. For the
            same reason \cmd{odgi extract} is $20\times$ faster than
            \cmd{vg chunk} extracting a portion from the full
            graph. \FIXME: vg never finished.
            %
            All timings were performed on a server class machine
            on SSD MODEL with X GB of RAM and 48 CPU cores (Y x 12 core
        AMD Opteron) Processor MODEL @ 3.3 GHz with 8MB L2
        Cache.\FIXME}
        \begin{tabular}{{@{}lllll@{}}}
          \toprule
          Operation & Command & Runtime & Memory  & File size     \\
                    &      & (mm:ss) & (GB) & (GB)           \\
          \midrule
          Convert GFA to native format                         &       &       &                   \\
          & \cmd{odgi build}      & 1:35  & 10.39 & 5.4               \\
          & \cmd{vg convert}      & 8:14  & 28.43 & 6.1               \\
          Extract subgraph                         &       &       &                   \\
          & \cmd{odgi extract}    & 1:15  & 9.43  & 2.7               \\
          & \cmd{vg chunk}        & 21:09 & 59.33 & 1.7               \\
          \botrule
        \end{tabular}
        \label{tab:02}

    \end{table}
