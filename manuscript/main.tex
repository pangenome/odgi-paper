\PassOptionsToPackage{utf8}{inputenc}
\documentclass{bioinfo}

\usepackage[draft]{hyperref}
\usepackage{makecell}
\usepackage{comment}

% singlelinecheck=false puts subcaptions on the left
\usepackage[singlelinecheck=false]{subcaption}

\usepackage{algorithm2e}
\usepackage[usenames,dvipsnames]{xcolor}

% we squeeze our figures even more together
\captionsetup{belowskip=-2pt}

\SetAlgoLined
\SetKwProg{MyStruct}{Struct}{ contains}{end}

\newcommand{\vocab}{\textbf}
\newcommand{\red}[1]{{\textcolor{Red}{#1}}}
\newcommand{\FIXME}[1]{\red{[FIXME: #1]}}

\def\labelitemi{--}

\copyrightyear{XXXX} \pubyear{XXXX}

\access{Advance Access Publication Date: Day Month Year}
\appnotes{Genome Analysis}

\begin{document}
\firstpage{1}

\subtitle{Genome Analysis}

\title[ODGI: understanding pangenome graphs]{ODGI: understanding pangenome graphs}
\author[Guarracino, Heumos \textit{et~al}.]{
Andrea~Guarracino\,$^{\text{\sfb 1} \dagger}$,
Simon~Heumos\,$^{\text{\sfb 2} \dagger}$,
Sven~Nahnsen\,$^{\text{\sfb 2}}$,
Pjotr~Prins\,$^{\text{\sfb 3}}$,
and~Erik~Garrison\,$^{\text{\sfb 3}*}$
}

\address{
$^{\text{\sf 1}}$University of Tor Vergata, Rome, Italy \\
$^{\text{\sf 2}}$Quantitative Biology Center (QBiC), University of T\"ubingen, T\"ubingen, Germany, 72076 \\
$^{\text{\sf 3}}$University of Tennessee Health Science Center, Memphis, TN, USA
}

\corresp{$^\ast$To whom correspondence should be addressed. \\
$^\dagger$Contributed equally.}

\history{Received on XXXXX; revised on XXXXX; accepted on XXXXX}

\editor{Associate Editor: XXXXXXX}

% XXX key message of the paper is that we have collected a set of algorithms that enable easy use of pangenome graphs for investigating biology

\abstract{
\textbf{Motivation:}
Pangenome graphs provide a complete representation of the mutual alignment of collections of genomes.
These models offer the opportunity to study the entire genomic diversity of a population, including structurally complex regions.
Nevertheless, analyzing hundreds of gigabase-scale genomes using pangenome graphs is difficult as it is not well-supported by existing tools.
Hence, fast and versatile software is required to ask advanced questions to such data in an efficient way. \\
\textbf{Results:}
We wrote ODGI, a novel suite of tools that implements scalable algorithms and has an efficient in-memory representation of DNA variation graphs.
ODGI includes tools for detecting complex regions, extracting \textit{loci}, removing artifacts, exploratory analysis, manipulation, validation, and visualization.
Its fast parallel execution facilitates routine pangenomic tasks as well as pipelines that can quickly answer complex biological questions of gigabase-scale pangenome graphs. \\
\textbf{Availability:}
ODGI is published as free software under the MIT open source license.
Source code can be downloaded from \url{https://github.com/pangenome/odgi} and documentation is available at \url{https://odgi.readthedocs.io}.
ODGI can be installed via Bioconda \url{https://bioconda.github.io/recipes/odgi/README.html} or GNU Guix \url{https://github.com/ekg/guix-genomics/blob/master/odgi.scm}. \\
\textbf{Contact:} \href{egarris5@uthsc.edu}{egarris5@uthsc.edu} \\
\textbf{Supplementary information:} Supplementary data are available at \textit{Bioinformatics} online.
}

\maketitle

\section{Introduction}
A pangenome models the full set of genomic elements in a given species or clade~\citep{cpang2018,Eizenga_2020}.
In contrast to reference-based approaches which relate sequences to a single genome, these data structures encode the mutual relationships between all the genomes represented.
In pangenome graphs~\citep{Paten:2017}, homologous regions between genomes are compressed into a single representative of all alleles present in the pangenome.
These flexible models let us encode any kind of variation, allowing the generation of comprehensive data systems on which to base analyses of genome evolution.
Although these data structures are of clear utility to researchers~\citep{cpang2018}, and have been the focus of numerous applications in read mapping~\citep{Garrison:2018,Baaijens_2019,Hickey:2020,Sibbesen_2021}, the scientific community still lacks a toolset specifically focused on graph manipulation and interrogation.

The Human Pangenome Reference Consortium (HPRC) and Telomere-to-Telomere (T2T) consortium~\citep{Miga:2020, Logsdon_2021, Nurk_2021} have recently demonstrated that high-quality \textit{de novo} assemblies can be routinely generated from third-generation long read sequencing data.
We anticipate that \textit{de novo} assemblies of similar quality will become common, leading to demand for methods that allow us to create and explore pangenomes.
Pangenome graphs are a natural model to support these kinds of analyses, but studies based on pangenome graphs are difficult due to a lack of downstream tools and standard workflows.

Here, we present the Optimized Dynamic Genome/Graph Implementation (ODGI) toolkit, a pangenome graph interrogation and transformation system specifically implemented to handle the data scales encountered when working with pangenomes built from hundreds of haplotype-resolved genomes.
ODGI provides a set of standard operations on the variation graph data model, generalizing ``genome arithmetic'' concepts like those found in BEDTools~\citep{Quinlan_2010} to work on pangenome graphs, and providing a variety of operations, such as visualization, sorting, and liftover projections, all critical to understand and exploit pangenome graphs.

%Tools in ODGI operate on an efficient dynamic HandleGraph model \citep{Eizenga_2020_BX}.
%Algorithms written against this abstract API can be applied to the graph.% applies algorithms based on the HandleGraph API.
%This common API lets us reuse and and extend algorithms shared with the VG toolkit (VG) \citep{Garrison:2018}.
%We specifically develop new methods focused on problems encountered when building pangenome graphs at the scale of vertebrate populations.
%To ease interactive use, the majority of the ODGI tools are implemented in an index-free manner, avoiding the need to create index structures at each step of complex graph processing pipelines.
%Thanks to ODGI's efficient path representation, its tools can work with variation graphs with highly complex regions.
%This eliminates one of the major bottlenecks when working with very large and deep variation graphs, and allows researchers to build and understand graphs of previously-inaccessible complexity and scale.

%\footnote{\url{https://humanpangenome.org/} (accessed Oct 2021)}
%\footnote{\url{https://sites.google.com/ucsc.edu/t2tworkinggroup/home} (accessed Oct 2021)}.

\section{Model}
A pangenome graph is a sequence model that encodes the mutual alignment of many genomes~\citep{Garrison_2019_thesis,Eizenga_2020}.
In the variation graph, $V = (N, E, P)$, nodes $N = n_1\ldots n_{|N|}$ contain sequences of DNA.
Each node $n_i$ has an identifier $i$ and an implicit reverse complement $\bar{n_i}$, and a node strand $s$ corresponds to one such orientations.
Edges $E = e_1\ldots e_{|E|}$ represent ordered pairs of node strands: $e_i = ( s_a, s_b )$.
Paths $P = p_1\ldots p_{|P|}$ describe walks over node strands: $p_i = s_1 \ldots s_{|p_i|}$.
When used as a pangenome graph, $V$ expresses sequences, haplotypes, and annotations as paths.
%The utility of the variation graph model lies in its lossless representation of genomes and their alignment.
By containing both the sequences and information about their relative variations, the variation graph provides a complete and powerful foundation for many bioinformatic applications.

Pangenome graphs can be constructed by multiple sequence alignment~\citep{Lee_2002,Grasso_2004} or by transitively reducing an alignment between sequences to an equivalent, labeled sequence graph~\citep{Kehr_2014,Garrison_2019_thesis}.
Current methods to build these graphs are still under active development~\citep{Li:2020,Armstrong:2020,pggb}, but they have largely settled on a common data model, represented in the Graphical Fragment Assembly (GFA) format \citep{GFA}).
%\FIXME{may want to move GFA standard+VG to intro section 1}. AG: it is borderline, but it seems fine to have it here too.
This standardization supports the development of a reference set of tools that operate on the pangenome graph model.
Such an effort began with the VG toolkit~\citep{Garrison:2018}.
Here we refocus it with ODGI, a compatible, but independent set of algorithms focused on visualization, interrogation, and transformation of pangenome graphs.

%A more general approach is to build the graph from an alignment between sequences.
%We first transitively collapse characters in the input sequences that align together into single character in the output genome
%By projecting input sequences through this transformation, we obtain paths that losslessly encode the original sequences, but in the space of the graph \citep{Garrison_2019_thesis}.

\section{Implementation}
The ODGI toolkit builds on existing approaches to efficiently store and manipulate variation graphs \citep{Garrison:2018}.
Similar to other efficient libaries presenting the \textsc{HandleGraph} model \citep{Eizenga_2020_BX}, ODGI's implementation rests on three key properties which hold for most pangenome variation graphs:

\begin{enumerate}
\item They are relatively sparse, with low average node degree.
\item They can be sorted so that most edges go between nodes that are close together in the sort order.
\item Their embedded paths are locally similar to each other.
%We incorporate concepts first introduced in the dynamic version of the Graph BWT \citep{Siren:2020} (GBWT), a scalable implementation of the graph extension of the positional Burrows-Wheeler transform \citep{Durbin_2014}.
\end{enumerate}

These properties are used to build efficient dynamic variation graph data structures \citep{Siren:2020,Eizenga_2020_BX}.
Sparsity (1) allows us to encode edges $E$ using adjacency lists rather than matrices or hash tables.
The local linear structure of the graph (2) lets us assign node identifiers that increase along the linear components of the graph, which supports a compact storage of edges and path steps as relativistic (usually small) differences rather than absolute (always large) integer identifiers.
Path similarity (3) allows us to write local compressors that reduce the storage cost of collections of path steps.
%\FIXME{without a figure we may lose the reader here a little} -- see algorithm 1

%Here, we build on this work and focus on issues that have arisen during our recent work on the Human Pangenome.
ODGI improves on prior efforts, based on issues that arose during our work with high-quality assemblies that cover almost all parts of the genome \citep{Logsdon_2021,Nurk_2021}.
%A key design objective, motivated by experience working with graphs built from high quality eukaryotic genome assemblies, is to support graphs with very high path depth.
For example, we find that it is necessary to support graphs that have regions of very high path depth.
Such motifs can occur in collapsed repeat structures generated by ambiguous sequence homology relationships in repeats found in the centromeres and other segmental duplications.
If we cannot process such regions, we have only two options: 1) we can remove these regions, or 2) we can leave them unaligned. However, neither of these solutions allows us to investigate their biological features.
To seamlessy represent such difficult regions, we follow an approach implemented in the dynamic version of the Graph BWT (GBWT) \citep{Siren:2020}, and built a node-centric, dynamic, compressed model of the paths.
This design supports node-local modification and update of the graph, which lets us build and transform the model in parallel.

We store the graph in a vector of node structures, each of which presents a node-local view of the graph sequence, topology, and path layout.
Expressed in terms of the variation graph $V$, ODGI's core $Node$ structure includes a decoder that maps the neighbors of each node to a dense range of integers.
For a given node $i$ and neighbor $j$, the decoder itself does not store the $id$ of $j$, but rather a compact representation of the relative difference between the node ids: $\delta = Node_i.id - Node_j.id$.
This keeps even the size of the encoding small, per common variation graph property (2).
We define the edges and path steps traversing the node in terms of this alphabet of $\delta$'s.
The structures in Algorithm \ref{alg:structs} describes our encoding.

\begin{algorithm}
\MyStruct{Node}{
    \textbf{id} $\in \mathbb{N}$ \tcp{an identifier}
    \textbf{lock} \tcp{atomic locking primitive}
    \textbf{sequence} $= [$A$|$T$|$G$|$C$|$N$]*$ \tcp{DNA}
    \tcp{bit-packed vector of edges}
    \textbf{edges} $= (x_i,x_j)* : (i, j) \in [1\ldots \Sigma]^2$ \\
    \tcp{bit-packed vector of id deltas}
    \textbf{decoding} $x_1 \ldots x_{\Sigma} \in \mathbb{N}^\Sigma$ \\
    \tcp{bit-packed vector of path steps}
    \textbf{path\_steps} $[Step_1 \ldots Step_n]*$
}
\MyStruct{Step}{
    \textbf{path\_id} $\in \mathbb{N}$ \tcp{the path's global id}
    \textbf{is\_rev} $\in ( 0, 1 )$ \tcp{the step orientation}
    \textbf{is\_start} $\in ( 0, 1 )$ \tcp{if first step in path}
    \textbf{is\_end} $\in ( 0, 1 )$ \tcp{if last step in path}
    \textbf{prev\_$\delta$} $\in [1\ldots \Sigma]$ \tcp{$\delta$-encoded previous node}
    \textbf{prev\_rank} $\in \mathbb{N}$ \tcp{step rank on previous node}
    \textbf{next\_$\delta$} $\in [1\ldots \Sigma]$ \tcp{$\delta$-encoded previous node}
    \textbf{next\_rank} $\in \mathbb{N}$ \tcp{step rank on next node}
}
\caption{ODGI's relativistically-packed $Node$ structure and the $Step$ structure used to represent the paths as doubly-linked lists.}
\label{alg:structs}
\end{algorithm}


Each structure contains the sequence of the node ($Node_i.sequence$), its edges in both directions ($Node_i.edges$), and a vector of path steps that describes the previous and next steps in paths that walk across the node ($Node_i.path\_steps$).
%As in the GBWT, to compress the encoding, we encode path steps using a local alphabet that maps the N neighbours of the node into the range $1\ldots N$.
%To further save space, node deltas, rather than IDs, are stored in this alphabet.
%The delta between two nodes is defined as their distance in the graph vector (i.e., the difference between the node offsets).
For efficiency, $Node_i.sequence$ is stored as a plain string, while the $edges$ and $path\_steps$ are stored using a dynamic succinct integer vector that requires $O(2nw)$ bits for the edges and $O(5nw)$ bits for the path steps, where $n$ is the number of steps on the node, and $w$ is $\approx log_2(n)$ \citep{prezza2017framework}.

To allow edit operations in parallel, each node structure includes a byte-width mutex $lock$.
All edit operations on the graph must touch at most two $Node$ structs at a time (both edge and path step representations are doubly-linked).
To avoid deadlocks, we acquire the node locks in ascending $Node.id$ order and release them in desceding order.
In addition to node-local features of the graph, we must maintain some global information.
Specifically, we record the start and end of paths, as well as a name to path id mapping in lock-free hashtables.
The use of lock-free hashtables lets us avoid a global lock when looking up path or graph metadata, which would quickly become a bottleneck during parallel operations on the graph.
By avoiding global locks, we implement many of the operations in ODGI using maximum parallelism available.
This ``atomic'' approach to edit operations on the graph is key to enable our methods to scale to the largest pangenome graphs that we can currently build.

\begin{comment}
%\subsection{Core functionality}

In the variation graph model, paths have to respect the graph’s topology: this can be verified with odgi validate, to ensure no errors in the input or edited graphs.
In variation graphs the coordinates are provided by the embedded path sequences.
Indeed, the node IDs are not meant to be stable. odgi position finds, translates, and liftovers graph and path positions between different graphs by exploiting their shared path sequences (Figure 1.B).
\end{comment}

\begin{comment}
key message of the paper is that we have collected a set of algorithms that enable easy use of pangenome graphs for investigating biology
-> build model solves problem of working with big graphs in memory
-> view (convert to GFA) & paths solve problem of exporting basic features of the graph (e.g. paths)
-> stats (understand basic size / structure) & bin & degree & depth solves problem of understanding the overall structure and size of the graph
-> sort (groom) & layout solves problem of finding latent structure in the pangenome
-> viz & draw provides a human-viewable readout of the graph
-> chop & unchop & squeeze & break & prune & explode lets us break apart or combine the graph nodes and topology
-> position & tips & untangle (jaccard based coordinate conversion) provides a way to map coordinates between any genomes in the graph (e.g. liftover!)
-> extract lets us pull out specific regions of the graph based on path ranges, nodes and positions
\end{comment}

\begin{figure*}[ht!]
  %\includegraphics[width=\linewidth,trim=+.225cm 0 +.425cm +2cm]{fig/metrics/chr4_HTT_chm13_degree_w1_bed.pdf}  
  \includegraphics[width=1.0\linewidth, trim=-0cm 2cm 0 0cm]{fig/odgi_tools.pdf}
  \caption{Methods provided by ODGI (in black) and their supported input (in blue) and output (in red) data formats.}
  \label{fig:operations}
\end{figure*}


\section{Results}

ODGI provides a set of first-order interrogative and manipulative operations on pangenome graphs.
We have established these tools to support our exploration of graphs built from hundreds of large eukaryotic genomes.
They are practical and able to work with high levels of graph complexity, even with regions of very high path depth (10$^5$ to 10$^6$-fold coverage).
ODGI's tools cover common operations that we have found to be essential when working with these complex biological models:

\begin{itemize}
\item \textit{odgi build} constructs the ODGI data model from GFA (\S\ref{sec:build}).
\item \textit{odgi view} converts the graph into standard text formats (\S\ref{sec:text}).
\item \textit{odgi matrix} derives the pangenome matrix (\S\ref{sec:text}).
\item \textit{odgi paths} lists and extracts paths in FASTA (\S\ref{sec:text}).
\item \textit{odgi flatten} converts the graph to FASTA and BED (\S\ref{sec:text}).
\item \textit{odgi viz} provides a scalable linear visualization of the graph (\S\ref{sec:viz}).
\item \textit{odgi draw} renders a 2D image of the graph (\S\ref{sec:viz}).
\item \textit{odgi extract} subsets the graph based on path ranges (\S\ref{sec:extract}).
\item \textit{odgi chop} breaks long nodes into shorter ones (\S\ref{sec:edit}).
\item \textit{odgi unchop} combines redundant nodes (\S\ref{sec:edit}).
\item \textit{odgi explode} breaks the graph into connected components (\S\ref{sec:edit}).
\item \textit{odgi squeeze} unifies disjoint graphs (\S\ref{sec:edit}).
\item \textit{odgi prune} removes complex regions (\S\ref{sec:edit}).
\item \textit{odgi position} lift coordinates between paths and graph positions (\S\ref{sec:untangle}).
\item \textit{odgi untangle} deconvolutes paths relative to a reference (\S\ref{sec:untangle}).
\item \textit{odgi tips} finds path end points relative to a reference (\S\ref{sec:untangle}).
\item \textit{odgi sort} orders the graph with sorting pipelines (\S\ref{sec:sort}).
\item \textit{odgi layout} establishes a 2D layout using SGD (\S\ref{sec:sort}).
\item \textit{odgi stats} provides numerical properties of the graph (\S\ref{sec:metrics}).
\item \textit{odgi bin} generates a summarized view of the graph (\S\ref{sec:metrics}).
\item \textit{odgi depth} describes path depth over graph and path positions (\S\ref{sec:metrics}).
\item \textit{odgi degree} describes node degree over graph and path positions (\S\ref{sec:metrics}).
\end{itemize}

Each tool focuses on a small set of related operations.
Most read or write the native ODGI format (`og' extension) (Figure \ref{fig:operations}) and work with standard text based data formats common to bioinformatics.
This supports the implementation of flexible, composable, graph processing pipelines based on a common data type.
While the pangenome graph is a first-class entity in ODGI, we focus on the embedded paths to provide a universal coordinate system.
By considering all paths in the graph as potential reference or query sequences, we make the graph invisible to downstream tools that operate on collections of sequences (e.g. BEDTools, SAMtools \citep{Li2009}, aligners).
This approach benefits from the information in the graph without strongly embedding our methods in this difficult new research context.

%We begin with a set of HandleGraph-based algorithms first establish in the VG toolkit \citep{Garrison:2018}.
%These include algorithms for graph traversal, partition-finding, $k$-mer and character enumeration, sorting, and pruning.
%Unlike VG, we provide no methods to construct graphs, map reads to the graph, or derive variants.

%We find that, to understanding relationships between sequences and positions in pangenome graphs, we must use context mapping to ``untangle'' complex regions found in VNTRs, segmental duplications, and centromeric repeats.
%This approach lets use pangenome graphs as a foundation for universal liftover between any two sets of sequences in the graph.

%Most of the tools are designed to be applied together, piping the output of one tool into the next, thereby preventing the creation of intermediate files, and reducing the number of IO operations.


\subsection{Building the \textsc{ODGI} model}
\label{sec:build}

We begin by transforming the storage model of the GFA format (in which nodes, edges, and paths are described independently) into the ODGI node-centric encoding with \textit{odgi build}.
This construction step can present a significant bottleneck, in particular as the size of the path set of the graph increases.
The ODGI data structure (Algorithm \ref{alg:structs}) allows algorithms that build and modify the graph to operate in parallel, without any global locks.
In \textit{odgi build}, we initially construct the node vector in a serial operation that scans across the input GFA. Then, in serial, we add edges in the $Node.edges$ vectors of pairs of nodes. Finally, we create paths in serial, and extend them in parallel by obtaining mutex $Node.lock$ for pairs of nodes and by adding the path step in their $Node.path\_steps$ vectors.
This parallelism speeds ODGI model construction by many-fold when testing against graphs built by the HPRC (Figure \ref{fig:build}).

%Figure: encoding model as a cartoon --  HTT exon 1 as tiny example graph (maybe)

\subsection{Converting to standard file formats}
\label{sec:text}

% pjotr & andrea
ODGI maintains its own efficient binary format for storing graphs on disk.
ODGI also supports reading and writing the textual GFA file format for interchange with other tools such as with the VG toolkit~\citep{Garrison:2018}.

\textit{odgi view} can convert a graph in ODGI binary format to the standard GFAv1~\citep{GFA} format. It can reveal a graph’s internal structures for e.g. debugging processes.

\textit{odgi stats} outputs pangenome statistics in TSV or YAML textual file formats.
Among other metrics, it can calculate the number of nodes, edges, paths and the total nucleotide length of the graph. {\color{red} FIXME: All this information is also listed in the metrics section. How to split? No mentioning of file formats in the following sections?}

\textit{odgi matrix} derives the pangenome matrix and writes the graph topology in sparse matrix formats.

\textit{odgi paths} allows the investigation of paths of a given variation graph. It can calculate overlap statistics of groupings of paths and extracts FASTA.

\textit{odgi flatten} projects the graph sequence and paths into FASTA and BED.

\textit{odgi bin} generates a summarized view of gigabase scale graphs providing tab delimited or JSON `bins'.

\textit{odgi position} translates graph and path positions between or within graphs emitting the liftovers in TSV format.

\textit{odgi server} implements a basic HTTP server which can translate a 'path:position' to a 'pangenome:position' very efficiently.

\textit{odgi degree/position/depth/untangle/tips} all can output the BED format.

\textit{odgi untangle} outputs the Pairwise mApping Format (PAF). PAF is a text format describing the approximate mapping positions between two set of sequences.
The odgi untangle command projects paths into a reference-relative BEDPE file, decomposing paralogy relationships.
During this process, it is capable of untangling loopy region resulting in linearized pairs of regions in the BEDPE file. A self dotplot assists in debugging and understanding the untangle process.


\subsection{Visualizing pangenome graphs}
\label{sec:viz}

% interesting that we don't go base-level at all... we use vg for that, or STM

Pangenome graph visualization is one of the first steps to gain insight into the mutual relationship between the sequences and their variation.
We pursue a novel approach to visualization with \textit{odgi viz} and \textit{odgi draw}, two tools which provide scalable ways of generating pictures of the high-level structure of the graph.

\textit{odgi viz} supports a binned, linearized rendering in 1 dimension (that is, all graph nodes lie on the same axis).
This visualization is computed in linear-time and offers a human-interpretable format suitable for understanding the topology and genome relationships in the pangenome graph (Fig. \ref{fig:odgi_viz}).
Graph nodes are arranged on a single axis, from left to right, with the colored bar indicating the paths and the nodes they cross.
White spaces indicate where paths do not traverse the nodes.
The meaning of the colors depends on how \textit{odgi viz} is executed.
By default, path colors has no particular meaning, but each path has its own color (Fig. \ref{fig:odgi_viz}\textbf{b}).
Path names are displayed on the left of the paths.
The black lines on the bottom indicate the edges connecting the nodes and, therefore, represent the graph topology.

Nevertheless, complex, nonlinear graph structures are difficult to display and interpret in a low number of dimensions.
To overcome such a limitation, \textit{odgi viz} supports multiple visualization modalities (Fig. \ref{fig:odgi_viz}\textbf{c-e}), making it easy to grasp the properties of the graph and its complexity.
Graph nodes order can affects downstream analyses on pangenome graphs. With \textit{odgi viz} we can color the paths by path position (Fig. \ref{fig:odgi_viz}\textbf{c}), with light grey indicating where paths begin and dark grey where they ends.
This visualization is suitable for checking graph nodes' linear order, as smooth color gradients indicate that the graph reflects well the linear paths' coordinate system.
Pangenome graphs represents both strand of DNA sequences.
\textit{odgi viz} supports also coloring the paths by orientation, with paths colored in red and black where their sequence is reversed or in forward, respectively, with respect to the sequences represented by the graph nodes (Fig. \ref{fig:odgi_viz}\textbf{d}).
Eukaryotic genomes experience gains and losses of genetic material, resulting in copy number variation (CNV) across the population.
With \textit{odgi viz}, we can use multiple color palettes to color the paths by path depth, highlighting the different copy number statuses in the genomes represented in the pangenome graph (Fig. \ref{fig:odgi_viz}\textbf{e}).


\textit{odgi draw} extends the visualization in 2 dimensions (2D)  (Fig. \ref{fig:odgi_viz}\textbf{a}) by rendering the layout built by \textit{odgi layout} (\S\ref{sec:sort}). A 2D rendering is more costly to compute, but we similarly provide a linear-time scaling implementation, allowing us to apply it to large pangenome graphs.


\subsection{Extracting regions of interest}
\label{sec:extract}

% andrea (?)

% extract
Pangenome graphs are very large, but we often only want to work with a small portion (e.g. a single gene).
We can extract such regions using coordinates on the paths in the graph to guide us.

Figure showing extraction process---either schematic or ``real''.

odgi extract allows extracting specific subregions of the graph as defined by query criteria, thereby simplifying the downstream analyses and reducing the resources to work only with the extracted region.


%Subsetting operations allow us to zoom in on particular genes or regions of interest for more precise examination.

%We extract key information about the graph using its paths as a pangenome reference system.

\subsection{Editing the graph structure}
\label{sec:edit}

% andrea (?)

% chop / unchop / squeeze / break / prune / explode

To enable efficient sequence alignment against the graph, long nodes can be divided into shorter nodes at a maximum requested size using \textit{odgi chop}. Partial order alignment, which consists of aligning sequences against a directed acyclic graph (DAG), is frequently used in pangenome building pipelines, but the current implementations return DAGs with 1-bp long nodes; \textit{odgi unchop} allows joining nodes that can be merged without changing the graph topology, nor the embedded sequences, obtaining an equivalent, but more compact, representation of the graph.

Pangenome graphs can embed multiple chromosomes as separated connected components (inter-chromosomal structural variants would join the components into bigger ones).
\textit{odgi explode} separates the connected components in different ODGI format files, while \textit{odgi squeeze} allows merging multiple graphs into the same ODGI format file, preventing node ID collisions.

Cycles in the graph complicate downstream analyses: \textit{odgi break} removes the cycles, reducing the complexity of the graph topology.
\textit{odgi groom} removes spurious inverting links by exploring the graph from the orientation supported by most paths.



These are commonly-needed basic operations on the topology of the graph.



\subsection{Untangling and navigating the pangenome with path-based coordinates}
\label{sec:untangle}

% erik

The key data in a pangenome graph is a representation of the alignment (or homology) relationships between sequences in the pangenome.
Navigating and understanding the graph requires coordinate systems that we can use to link other data into the graph model, and thus to all genomes in the pangenome.
Primarily, ODGI uses the embedded pangenome sequences to provide a set of coordinates in a standard, 1D format.
These coordinates depend on the sequences, and are graph-independent.
Our for positional conversion (\textit{odgi position}), annotation liftover (\textit{odgi untangle}), breakpoint identification (\textit{odgi tips}), subgraph extraction (\textit{odgi extract}), visualization (\textit{odgi viz}), path depth (\textit{odgi depth}), and graph complexity measurement (\textit{odgi degree}) can all operate on coordinates of paths in the graph.

For understanding the large-scale organization of the graph, we benefit from an even simpler coordinate system.
To provide a universal coordinate space for the graph, we project its nodes into a low-dimensional space (in practice, 1 or 2 dimensions) by learning the projection via a HOGWILD! \citep{niu2011hogwild}, path-guided stochastic gradient descent (PG-SGD) layout algorithm that adapts SGD-based drawing to pangenome graphs \citep{zheng2018graph}.
This approach updates node relative positions to best-match their distance in the paths running through the graph.
Based on this projection, we can trivially generate 1D and 2D visualizations.
We can also detect regions whose layout is distorted, and which thus may represent structural variation or assembly error.
ODGI can project vector and matrix representations of the graph relative to these learned coordinate spaces.

In addition to these projections, we use the paths in the graph to provide a universal coordinate space.
It also supports several kinds of ``lift-over'' of coordinates between different genomes (Fig. \ref{fig:odgi_viz}\textbf{f-g}).
These allow for the direct translation of single coordinates and ranges, in an interactive way.
And, they let us extract a pairwise 1:many alignment between a given set of ``query'' sequences and a given set of ``target'' or reference sequences.
This lets us convert the graph to lift-over maps compatible with standard software for projecting annotations and alignments from one genome to another.



% position / tips
% untangle (this follows on the jaccard graph mapping concepts in the previous section}
%The graph is a model of an alignment of many genomes.
%We can use coordinates in any genome to refer to it, but this requires a few basic operations ...
%Obtaining unambiguous mappings between different genomes requires the use of a new kind of graph based mapping (path graph jaccard).


To obtain a more precise overview of collapsed \textit{loci}, we can apply \textit{odgi untangle} to segment paths into linear segments by breaking these segments where the paths loop back on themselves.
In this way, we can obtain base-level information on the copy number status of the sequences in the locus.

Figure showing the untangling of C4 (Fig. \ref{fig:odgi_viz}\textbf{h})
Figure showing tips over a single human chromosome from HPRC.

\subsection{Sorting the pangenome graph topology}
\label{sec:sort}

% simon

% sort / layout / NO tension! We keep that for the sorting paper

% Allows us to understand the sparse structures typically found in pangenome graphs.

%Pangenome graphs built from raw sets of alignments may have complex structures which can introduce difficulty in downstream analyses. Uncovering their latent topology allows us to understand their typical sparse structures.

\textit{odgi sort} offers multiple sorting algorithms to find the best node order in one dimension. Most notably, nodes can be sorted topologically, randomly, by breaking cycles in the graph, by grooming, or by using a novel path-guided stochastic gradient descent (PG-SGD) algorithm.
The latter one exploits the biological information in the paths for sorting the nodes.
Combining a series of sorts can mitigate the weaknesses of individually applied sorting algorithms.
%In Figure \textbf{\textit{TODO MHC wgg.85}}, a comparison of an unsorted human pangenome MHC graph and a sorted one are displayed.

%For the establishment of a layout in two dimensions \textit{odgi layout} applies a 2D adjusted PG-SGD algorithm. A calculated layout can be visualized statically with \textit{odgi draw} (see Figure \textit{\textbf{TODO MHC wgg.85}}) or interactively explored with the in-development \textit{gfaestus}\{\url{https://github.com/chfi/gfaestus}}.

\subsection{Obtaining metrics of the pangenome graph}
\label{sec:metrics}

% HTT exon1

% stats / multiQC / bin / degree / depth

Graphs statistics provide alternative ways to gain insight into pangenomes complexity revealing the overall structure, size, and features of a graph and its sequences.

Applying \textit{odgi stats}, users can retrieve metrics describing the graph properties, such as the number of nodes, edges, paths, and graph length. It outputs pangenome statistics in TSV or YAML textual file formats.
MultiQC's \citep{Ewels_2016} ODGI module %\footnote{\url{https://multiqc.info/docs/\#odgi} (accessed Oct 2021)}
provides an interactive way to comparatively explore the statistics of an arbitrary number of graphs.
ODGI also offers more advanced tools for the interrogation of the graphs. \textit{odgi bin} summarizes the path information into bins of a specified size, enabling a summarized view of gigabase scale graphs. It provides tab delimited or JSON `bins'.
%This concept found its way into the pangenome graph ontology \cite{Yokoyama2020}. %maybe we want to save this up for the discussion
\textit{odgi depth} returns the node depth as defined by query criteria, allowing users to assess the complexity of regions in the graph.
Collapsed sequences with highly identical repeats indicate intricate loci.
Complex motifs can be also detected with \textit{odgi degree}, which returns the node degree as defined by query criteria.
High degree nodes can be the mirror of variation, misassemblies, or problems in the pangenome building, making the tool useful for validation.
In Fig. \ref{fig:metrics}, the characteristics of a 90 haplotypes human pangenome graph of the exon 1 huntingtin gene \citep{Sathasivam2013,Neueder2017} (\textit{HTTexon1}), obtained with the ODGI metrics tools, are shown.

\begin{figure}[ht!]
%  \centering
\floatbox[{\capbeside\thisfloatsetup{capbesideposition={left,top},capbesidewidth=0.45\linewidth}}]{figure}[\FBwidth]
         {\caption{Performance evaluation of \textit{odgi build} when translating a 90-haplotype graph of human chromosome 6 into ODGI's native format.
             Build time decreases as parallelism increases, but with diminishing per-thread processing rates.
           }\label{fig:build}}
         {\includegraphics[width=\linewidth]{fig/build/mean_build_time.pdf}} %\includegraphics[width=5cm]{name}}
	     %\includegraphics[width=\linewidth]{fig/build/mean_build_time.pdf}
	     %\caption{Performance evaluation of \textit{odgi build} when translating a 90-haplotype graph of human chromosome 6 into ODGI's native format.}
	
\end{figure}


\begin{figure*}[h!]
    \begin{subfigure}{\linewidth}
        \caption{}
        \centering
        % include first image
        \includegraphics[width=1.0\linewidth, trim=0 +2cm 0 0]{fig/visualization_1D/chr6.pan.fa.c3d3224.7748b33.395c7f4.smooth.gfa.C4.sorted}
        \label{fig:odgi_viz_default}
    \end{subfigure}
    \begin{subfigure}{\linewidth}
        \caption{}
        \centering
        % include second image
        \includegraphics[width=1.0\linewidth, trim=0 +2cm 0 0]{fig/visualization_1D/chr6.pan.fa.c3d3224.7748b33.395c7f4.smooth.gfa.C4.sorted.du}
        \label{fig:odgi_viz_color_by_path_pos}
    \end{subfigure}
    \begin{subfigure}{\linewidth}
        \caption{}
        \centering
        % include second image
        \includegraphics[width=\linewidth, trim=0 +2cm 0 0]{fig/visualization_1D/chr6.pan.fa.c3d3224.7748b33.395c7f4.smooth.gfa.C4.sorted.z}
        \label{fig:odgi_viz_color_by_inversion_rate}
    \end{subfigure}
    \begin{subfigure}{1\linewidth}
        \caption{}
        \centering
        % include fourth image
        \includegraphics[width=\linewidth, trim=0 1.5cm 0 0]{fig/visualization_1D/chr6.pan.fa.c3d3224.7748b33.395c7f4.smooth.gfa.C4.sorted.m}
        \label{fig:odgi_viz_color_by_path_depth}
    \end{subfigure}
    \caption{
        Visualizing the complement component 4 (C4) pangenome graph extracted from a whole human pangenome graph of 90 haplotypes.
        In all visualizations, 10 paths are displayed: 2 reference genomes (chm13 and grch38 on the top) and 8 haplotypes from 4 individuals.
        \textbf{(a)} default modality: the image shows a quite linear graph.
        The 2 longer links at the bottom indicate the presence of a structural variant (long link) with another structural variant nested inside it (short link).
        Indeed, human C4 exists as 2 functionally distinct genes, C4A and C4B, which both vary in structure and copy number \citep{Sekar_2016}. The longer link indicates that the copy number status varies across the haplotypes represented in the pangenome.
        Moreover, C4A and C4B genes segregate in both long and short genomic forms, distinguished by the presence or absence of a human endogenous retroviral (HERV) sequence, as also highlighted by the short nested link on the bottom.
        \textbf{(b)} color by path position: the color gradients are smooths, highlighting that its node are well sorted in 1 dimension.
        The top two reference genomes and 2 haplotypes (with names starting with HG01978\#1 and HG03492\#2) go from left to right, while 5 haplotypes go in the opposite direction, as indicated by the black color on their left.
        \textbf{(c)} color by strandness: the red paths indicate the haplotypes that were assembled in reverse with respect to the 2 reference genomes.
        \textbf{(d)} color by path depth: using the Spectra color palette with 4 level of path depths, white indicates no depth, while grey, red, and yellow indicate depth 1, 2, and 3, respectively.
        Coloring by path depth, we can see that the two references present two different allele copies of the C4 genes, both of them including the HERV sequence. The entirely grey paths have one copy of these genes.
        The path with name starting with HG01071\#2 presents 3 copies of the genes (indicated by the orange color), of which only 1 with the HERV sequence (gray color in the middle of the orange region).
    }
    \label{fig:odgi_viz}
\end{figure*}


\begin{figure*}[h!]
	\begin{subfigure}{\linewidth}
		\caption{}
		\centering
		% include first image
		\includegraphics[width=1.0\linewidth]{fig/metrics/chr4_pan_HTTex1_gfa_multiqc_odgi_stats.png}  
		\label{fig:metrics-multiqc}
	\end{subfigure}

	\begin{subfigure}{\linewidth}
		\caption{}
		\centering
		% include second image
		\includegraphics[width=1.0\linewidth, trim=0 0 -1cm 0 ]{fig/metrics/chr4_pan_fa_a2fb268_e820cd3_9ea71d8_smooth_gfa_og_HTTex1_og_O_og_tiny_og_png_svg.pdf}  
		\label{fig:metrics-viz}
	\end{subfigure}
	\begin{subfigure}{\linewidth}
		\caption{}
		\centering
		% include second image
		\includegraphics[width=\linewidth,trim=+.225cm 0 +0.425cm +2cm]{fig/metrics/chr4_HTT_chm13_depth_w1_bed.pdf}  
		\label{fig:metrics-depth}
	\end{subfigure}
	\begin{subfigure}{1\linewidth}
		\caption{}
		\centering
		% include fourth image
		\includegraphics[width=\linewidth,trim=+.225cm 0 +.425cm +2cm]{fig/metrics/chr4_HTT_chm13_degree_w1_bed.pdf}  
		\label{fig:metrics-degree}
	\end{subfigure}
%	\includegraphics[width=\linewidth]{fig/metrics/chr4.pan.HTTex1.gfa.multiqc_odgi_stats.png}
	\caption{Features of a 90 haplotypes human pangenome graph of the exon 1 huntingtin gene (\textit{HTTexon1}): \textbf{(a)} Excerpt of vital statistics of the \textit{HTTexon1} graph displayed by MultiQC's ODGI module. The very high GC content of 73.0\% compared to a human genomic mean GC content of 40.9\% \cite{Piovesan2019} is in accordance with the literature (see for example \cite{Sathasivam2013, Neueder2017}). \textbf{(b)} \texttt{odgi viz} visualization of the 13 largest gene alleles, CHM13, and GRCH38 of the \textit{HTTexon1} graph. \textbf{(c)} Per nucleotide node depth distribution of CHM13 in the \textit{HTTexon1} graph. The alternating depth around position 200 indicates polymorphic variation. \textbf{(d)} Per nucleotide node degree distribution of CHM13 in the \textit{HTTexon1} graph. The polymorphic region around position 200 complements the above node depth analysis. Figures \textbf{(b)-(d)} clearly highlight the variant region around position 200 showing the variable number of glutamine residues of the different individuals as reported by \cite{Nance1999}.}
	\label{fig:metrics}
\end{figure*}



% Figure: small graph with all the stats shown. A screenshot of MultiQC, visualize summary per bin somehow, R visualizations of degree and depth HTT exon 1 as tiny example graph


\section{Discussion}

% erik & pjotr & anyone

Pangenome graphs stand to become a ubiquitous tool in genomics \citep{Eizenga_2020}. With ODGI we implemented a state-of-the-art tool suite that can transform, analyse, simplify, validate, and visualise pangenome graphs at large scale.
%
Lifting over annotations and linearizing nested graph structures place the suite as the bridge between traditional linear reference genome analysis and pangenome graphs. ODGI is a unique set of tools that enables scientists to explore and discover the underlying biology of pangenome graphs. Already, the tools are the backbone of pipelines such as the Pangenome Graph Builder \citep{pggb} (PGGB) or nf-core/pangenome \citep{pangenome}. %\footnote{\url{https://github.com/nf-core/pangenome} (accessed Oct 2021)}.
Future work will add support for RNA and protein sequences and expand on metadata capabilities of large pangenome graphs.

ODGI provides common operation on standard pangenomic file formats, such as GFA. ODGI aims to be a reference implementation for many pangenome related operations. In the near future we aim to improve RDF support for annotation in the in-memory graph~\citep{Yokoyama2020}, for example, and allow for federated queries on the pangenome, and FAIR data sharing.

ODGI's visualization tools produce static, large-scale snapshots. Recent interactive browsers are reference centric \citep{Beyer2019, Yokoyama2019, Durant2021, Liang2021} or focusing on one dimension \citep{Wick_2015, Gonnella2018}. However, an interactive visualization solution that combines the 1D and 2D layout of a graph with annotation and read mapping information across different zoom levels is still missing. ODGI's graph layout algorithms could become a backbone of such tools, as is already the case for \textit{gfaestus} \citep{gfaestus}%\footnote{\url{https://github.com/chfi/gfaestus}}, which currently relies on the \textit{odgi layout} output for its interactive 2D visualization.

So far, ODGI has not been applied to graphs build from scWGS \citep{Zhuo2021} data. Each cell would be represented as a path in the graph, the path depth would be very high. As ODGI was designed to deal with the path depth of human pangenome graphs, it remains to be seen, if it will work equally well with scWGS data.

With the increased adoption of long read sequencing we expect pangenome tools to become increasingly common in biomedical research. Particularly for targets that involve complex variation, such as cancer, plant genomics and metagenomics, ODGI should facilitate disentangling, describing and analysing a much larger set of variation than previously was possible with tools that depend on short reads and reference genomes.

\section*{Funding}

We gratefully acknowledge support from NIH/NIDA U01DA047638 (EG) and NIH/NIGMS R01GM123489 (EG and PP).
SH acknowledges funding from the Central Innovation Programme (ZIM) for SMEs of the Federal Ministry for Economic Affairs and Energy of Germany. SN acknowleges Germany’s Excellence Strategy (CMFI), EXC-2124 and (iFIT) - EXC 2180 – 390900677.
This work was supported by the BMBF-funded de.NBI Cloud within the German Network for Bioinformatics Infrastructure (de.NBI) (031A537B, 031A533A, 031A538A, 031A533B, 031A535A, 031A537C, 031A534A, 031A532B).

\section*{Data availability}

Data used to build Human pangenome graphs is available at \url{https://github.com/human-pangenomics/HPP_Year1_Data_Freeze_v1.0}.
TODO!

\bibliographystyle{natbib}
%\bibliographystyle{achemnat}
%\bibliographystyle{plainnat}
%\bibliographystyle{abbrv}
%\bibliographystyle{bioinformatics}
%
%\bibliographystyle{plain}
%
\bibliography{document}


% \begin{thebibliography}{}

% \bibitem[Bofelli {\it et~al}., 2000]{Boffelli03}
% Bofelli,F., Name2, Name3 (2003) Article title, {\it Journal Name}, {\bf 199}, 133-154.

% \bibitem[Bag {\it et~al}., 2001]{Bag01}
% Bag,M., Name2, Name3 (2001) Article title, {\it Journal Name}, {\bf 99}, 33-54.

% \bibitem[Yoo \textit{et~al}., 2003]{Yoo03}
% Yoo,M.S. \textit{et~al}. (2003) Oxidative stress regulated genes
% in nigral dopaminergic neurnol cell: correlation with the known
% pathology in Parkinson's disease. \textit{Brain Res. Mol. Brain
% Res.}, \textbf{110}(Suppl. 1), 76--84.

% \bibitem[Lehmann, 1986]{Leh86}
% Lehmann,E.L. (1986) Chapter title. \textit{Book Title}. Vol.~1, 2nd edn. Springer-Verlag, New York.

% \bibitem[Crenshaw and Jones, 2003]{Cre03}
% Crenshaw, B.,III, and Jones, W.B.,Jr (2003) The future of clinical
% cancer management: one tumor, one chip. \textit{Bioinformatics},
% doi:10.1093/bioinformatics/btn000.

% \bibitem[Auhtor \textit{et~al}. (2000)]{Aut00}
% Auhtor,A.B. \textit{et~al}. (2000) Chapter title. In Smith, A.C.
% (ed.), \textit{Book Title}, 2nd edn. Publisher, Location, Vol. 1, pp.
% ???--???.

% \bibitem[Bardet, 1920]{Bar20}
% Bardet, G. (1920) Sur un syndrome d'obesite infantile avec
% polydactylie et retinite pigmentaire (contribution a l'etude des
% formes cliniques de l'obesite hypophysaire). PhD Thesis, name of
% institution, Paris, France.

% \end{thebibliography}
\end{document}
